\section{El problema de los dos cuerpos}

\begin{frame}{El problema de los dos cuerpos}

    Si tenemos dos cuerpos que interaccionan mediante un potencial radial con simetría esférica, el hamiltoniano del sistema está dado por:
        
    \begin{equation*}
        \hat{H} (\hat{p}_{1}, \hat{p}_{2}, \Vec{r}_{1}, \Vec{r}_{2}) = \hat{H}_{1} (\hat{p}_{1}, \Vec{r}_{1} ) + \hat{H}_{2} ( \hat{p}_{2}, \Vec{r}_{2}) + V(\Vec{r}_{1}, \Vec{r}_{2})
    \end{equation*}
        
    La ecuación de Schr\"odinger, $E \Psi=\hat{H} \Psi $, toma la forma \cite{de2014introduccion}
    
    \begin{align*}
        E \Psi &= \left[ \hat{H}_{1}  + \hat{H}_{2}  + V(\Vec{r}_{1}, \Vec{r}_{2}) \right] \Psi \\
        &=  \left[ - \frac{\hbar^2}{2m_{1}} \nabla^{2}_{1} - \frac{\hbar^2}{2m_{2}} \nabla^{2}_{2}  + V_{1} (\Vec{r}_1)   + V_{2} (\Vec{r}_2)  + V(\Vec{r}_{1}, \Vec{r}_{2}) \right] \Psi
    \end{align*}
    
\end{frame}

\begin{frame}{}

    Suponiendo que la partículas están aisladas y solo interaccionan entre si, es decir $V_{1}=V_{2}=0$.

    \begin{align*}
        E \Psi &=  \left[ - \frac{\hbar^2}{2m_{1}} \nabla^{2}_{1} - \frac{\hbar^2}{2m_{2}} \nabla^{2}_{2} + V(\Vec{r}_{1}, \Vec{r}_{2}) \right] \Psi
    \end{align*}
    
    Consideramos potenciales que dependen solo de la posición relativa, es decir, $V(\Vec{r}_{1}, \Vec{r}_{2}) = V(\Vec{r}_{1} - \Vec{r}_{2})$.
    
    \begin{align*}
        E \Psi &=  \left[ - \frac{\hbar^2}{2m_{1}} \nabla^{2}_{1} - \frac{\hbar^2}{2m_{2}} \nabla^{2}_{2} + V(\Vec{r}_{1} - \Vec{r}_{2}) \right] \Psi
    \end{align*}
    
\end{frame}

\begin{frame}{}
    
    La función de onda está descrita por 6 variables independientes ($\Vec{r}_1$, $\Vec{r}_2$), podemos cambiar a otras 6 variables más convenientes.
    
    \begin{equation*}
        \Vec{r} = \Vec{r}_1 - \Vec{r}_2
    \end{equation*}
    
    \begin{equation*}
        \Vec{R} = \frac{ m_1 \Vec{r}_1 + m_2 \Vec{r}_2 }{M} 
        \quad \mathrm{;} \quad
        M = m_1 + m_2
    \end{equation*}
    
    De manera que
    
    \begin{equation*}
        \nabla_1 = \nabla_r + \frac{m_1}{M} \nabla_R
        \quad \mathrm{y} \quad
        \nabla_2 = - \nabla_r + \frac{m_2}{M} \nabla_R
    \end{equation*}
    
\end{frame}

\begin{frame}{}
    
    De esta manera
    
    \begin{align*}
        \nabla_1^2 &= \left( \nabla_r + \frac{m_1}{M} \nabla_R \right)
        \left( \nabla_r + \frac{m_1}{M} \nabla_R \right) \\
        &= \nabla_r^2 + \frac{m_1}{M} \nabla_r \nabla_R + \frac{m_1}{M} \nabla_R \nabla_r + \frac{m_1^2}{M^2} \nabla_R^2
    \end{align*}
    
    \begin{align*}
        \nabla_2^2 &= \left( - \nabla_r + \frac{m_2}{M} \nabla_R \right)
        \left( - \nabla_r + \frac{m_2}{M} \nabla_R \right) \\
        &= \nabla_r^2 - \frac{m_2}{M} \nabla_r \nabla_R - \frac{m_2}{M} \nabla_R \nabla_r + \frac{m_2^2}{M^2} \nabla_R^2
    \end{align*}
    
\end{frame}

\begin{frame}{}
    
    Sumando $ \nabla_1^2 / m_1 $ y $ \nabla_2^2 / m_2 $
    
    \begin{align*}
        \frac{\nabla_1^2}{m_1} + \frac{\nabla_2^2}{m_2}
        &= \frac{\nabla_r^2}{m_1} + \frac{m_1}{M^2} \nabla_R^2 + \frac{\nabla_r^2}{m_2} + \frac{m_2}{M^2} \nabla_R^2 \\
        &= \left( \frac{1}{m_1} + \frac{1}{m_2} \right) \nabla_r^2 + \left(  \frac{m_1}{M^2} + \frac{m_2}{M^2} \right) \nabla_R^2 \\
        &= \left( \frac{m_1 + m_2}{m_1 m_2}  \right) \nabla_r^2 + \frac{1}{M} \nabla_R^2 \\
        &= \frac{1}{ \mu}  \nabla_r^2 + \frac{1}{M} \nabla_R^2
    \end{align*}
    
    Donde se define $\mu = \frac{m_1 m_2}{m_1 + m_2}$ como la masa reducida del sistema.
    
\end{frame}

\begin{frame}{}
    
    Sustituyendo el anterior resultado en la ecuación de Schr\"odinger
    
    \begin{align*}
        E \Psi &=  \left[ - \frac{\hbar^2}{2M} \nabla^{2}_{R} - \frac{\hbar^2}{2 \mu } \nabla^{2}_{r} + V( \Vec{r} ) \right] \Psi
    \end{align*}
    
    Sea $ \Psi (\Vec{R}, \Vec{r}) = \Phi ( \Vec{R}) \psi (\Vec{r} ) $
    
    \begin{align*}
        E \Phi ( \Vec{R}) \psi (\Vec{r} ) &=  \left[ - \frac{\hbar^2}{2M} \nabla^{2}_{R} - \frac{\hbar^2}{2 \mu } \nabla^{2}_{r} + V( \Vec{r} ) \right] \Phi ( \Vec{R}) \psi (\Vec{r} ) 
    \end{align*}
    
    \begin{align*}
        E &= \frac{1}{\Phi} \left[ - \frac{\hbar^2}{2M} \nabla^{2}_{R} \Phi \right] + \frac{1}{\psi} \left[ - \frac{\hbar^2}{2 \mu } \nabla^{2}_{r} \psi + V( \Vec{r} ) \psi \right]
    \end{align*}
    
\end{frame}


\begin{frame}{}
    
    Definimos
    
    \begin{align*}
        f( \Phi ) = \frac{1}{\Phi} \left[ - \frac{\hbar^2}{2M} \nabla^{2}_{R} \Phi \right]
        \quad \mathrm{y} \quad
        g( \psi ) = \frac{1}{\psi} \left[ - \frac{\hbar^2}{2 \mu } \nabla^{2}_{r} \psi + V( \Vec{r} ) \psi \right]
    \end{align*}
    
    \begin{equation*}
        \implies E = f( \Phi ) +g( \psi )
    \end{equation*}
    
    \begin{equation*}
        \implies E - g( \psi ) = f( \Phi )
    \end{equation*}
    
    La ecuación anterior solo tiene sentido si ambos lados de la ecuación son una contante
    
    \begin{equation*}
        \implies E - g( \psi ) = f( \Phi ) \equiv E_R
    \end{equation*}
    
    \begin{equation*}
        \implies g( \psi ) = E - f( \Phi ) = E- E_R \equiv E_r
    \end{equation*}
    
\end{frame}

\begin{frame}{}
    
    Por lo tanto 
    
    \begin{align*}
        E_R = \frac{1}{\Phi} \left[ - \frac{\hbar^2}{2M} \nabla^{2}_{R} \Phi \right]
        \quad \mathrm{y} \quad
        E_r = \frac{1}{\psi} \left[ - \frac{\hbar^2}{2 \mu } \nabla^{2}_{r} \psi + V( \Vec{r} ) \psi \right]
    \end{align*}
    
    Es decir, tenemos la ecuaciones
    
    \begin{tcolorbox}[colback=blue!5!white, colframe =blue!75!black, title= Las ecuaciones del problema de dos cuerpos]
        \begin{align*}
        E_R \Phi ( \Vec{R} ) = - \frac{\hbar^2}{2M} \nabla^{2}_{R}  \Phi ( \Vec{R} )
        \end{align*}
        
        \begin{align*}
            E_r \psi ( \Vec{r} ) =  - \frac{\hbar^2}{2 \mu } \nabla^{2}_{r} \psi ( \Vec{r} ) + V( \Vec{r} ) \psi ( \Vec{r} )
        \end{align*}
    \end{tcolorbox}
    
\end{frame}

\begin{frame}{}

    La primera de las ecuaciones 
    
    \begin{align*}
        E_R \Phi ( \Vec{R} ) = - \frac{\hbar^2}{2M} \nabla^{2}_{R}  \Phi ( \Vec{R} )
    \end{align*}
        
    hace referencia al movimiento de de una cuasipartícula libre de masa $M$, cuya posición es el centro de masas de las partículas $m_1$ y $m_2$.
    
    Este movimiento es irrelevante y se puede tomar la solución trivial $E_R = 0$ y $\Phi (\Vec{R}) = cte$ de un marco de referencia en el centro de masas donde la cuasipartícula está en reposo.
    
\end{frame}

\begin{frame}{}

    La segunda ecuación
    
    \begin{align*}
        E_r \psi ( \Vec{r} ) =  - \frac{\hbar^2}{2 \mu } \nabla^{2}_{r} \psi ( \Vec{r} ) + V( \Vec{r} ) \psi ( \Vec{r} )
    \end{align*}
        
    hace referencia al movimiento de de una cuasipartícula libre de masa $\mu$, sujeta al pontencial $V(\Vec{r})$.
    
    \vspace{0.5cm}
    
    Esta es la ecuación de Schr\"odinger para el problema de dos partículas reducida a una partícula (de 6 variables a 3 variables).
    
\end{frame}

\begin{frame}{}

    Considerando potenciales radiales con simetría esférica $V(\Vec{r}) = V(r) $, la ecuación de Schr\"odinger nos queda 
    
    \begin{align*}
        E_r \psi ( r, \theta, \phi ) =  - \frac{\hbar^2}{2 \mu } \nabla^{2}_{r} \psi ( r, \theta, \phi ) + V( r) \psi ( r, \theta, \phi )
    \end{align*}

    Esta ecuación es bien conocida, proponiendo $\psi (\Vec{r}) = R(r) Y( \theta, \phi )$ y considerando $ \nabla^{2}_{r} $ en coordenadas esféricas se puede separar fácilmente la ecuación.
    
\end{frame}

\begin{frame}{}

    \begin{tcolorbox}[colback=blue!5!white, colframe =blue!75!black, title= Ecuaciones separadas \cite{griffiths2018introduction}]
    
        La ecuación radial es:
        
        \begin{align*}
            \frac{1}{R} \frac{d}{dr} \left( r^2 \frac{dR}{dr} \right) - \frac{2mr^2}{\hbar^2} \left[ V(r) - E \right] = l(l+1)
        \end{align*}
        
        la solución depende del potencial.
        
        La ecuación angular es:
        
        \begin{align*}
            \frac{1}{Y} \left\{ \frac{1}{\sin{\theta}} \frac{\partial}{\partial \theta} \left(  \sin{\theta} \frac{ \partial Y}{\partial \theta} \right) + \frac{1}{\sin^2{\theta}} \frac{\partial^2 Y }{ \partial \phi^2 } \right\} = -l(l+1)
        \end{align*}
        
        cuya solución es $Y_{l,m} ( \theta, \phi ) $ (los armónicos esféricos).
        
    \end{tcolorbox}
    
\end{frame}






