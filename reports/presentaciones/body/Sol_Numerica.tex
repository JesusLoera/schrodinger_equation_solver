\section{Solución numérica a la ecuación de Schr\"odinger}

\begin{frame}{Ecuación de Schr\"odinger en coordenadas esféricas}

Se propone una forma para la función de onda del tipo $\Psi=Y(\theta,\phi)R(r)$, donde $Y(\theta,\phi)$ es la misma para todos los potenciales esfericamente simétricos y está en términos de los \textit{armónicos esféricos}.\\

\vspace{5mm}
Hay que resolver $R(r):$
\begin{equation*}
    \frac{d}{dr}\left(r^{2}\frac{dR}{dr}\right) - \frac{2mr^{2}}{\hbar^{2}}[V(r)-E]R = l(l+1)R
\end{equation*}

Efectuando un cambio de variable $u(r)=rR(r)$ se obtiene:
\begin{equation*}
    -\frac{\hbar^{2}}{2m}\frac{d^{2}u}{dr^{2}} + \left[V+\frac{h^{2}}{2m}\frac{l(l+1)}{r^2}\right]u = Eu
\end{equation*}
    
\end{frame}

\begin{frame}{}

Dando una forma más adecuada a la ecuación:
\begin{equation*}
    \frac{d^{2}u_{l}(r)}{dr^{2}} + \frac{2\mu}{\hbar^{2}} \left[(E-V)-\frac{\hbar^{2}}{2\mu}\frac{l(l+1)}{r^{2}}\right]u_{l}(r) = 0
\end{equation*}

En donde se ha definido:\\
\vspace{5mm}
$\mu:=$ masa reducida $=\frac{m_{q}m_{\Bar{q}}}{m_{q}+m_{\Bar{q}}} = \frac{m_{q}^{2}}{2m_{q}} = \frac{m_{q}}{2}$\\
\vspace{5mm}
$m_{q}:=$ masa reducida del quark pesado\\
\vspace{5mm}
$u_{l}(r):= rR_{l}(r)$ (l fijo de $Y_{lm}(\theta,\phi)$) \\
\vspace{5mm}
$V(r):=$ Potencial de Cornell
    
\end{frame}

\begin{frame}{}

Si se define:
\begin{equation*}
    K_{l}(r) = \frac{2\mu}{\hbar^{2}} \left[(E-V(r)) - \frac{\hbar^{2}}{2\mu} \frac{l(l+1)}{r^{2}} \right]
\end{equation*}

Particularmente para l=0:
\begin{equation*}
    K_{0}(r) = \frac{2\mu}{\hbar^{2}} (E-V(r))
\end{equation*}

Entonces:
\begin{equation*}
    \frac{d^{2}u_{l}(r)}{dr^{2}} = -K_{l}(r)u_{l}(r)
\end{equation*}

\vspace{5mm}
Se busca resolver esta EDO y obtener los eigenvalores de la energía
    
\end{frame}

\begin{frame}{El método de Numerov}

El método de Numerov nos permite integrar ecuaciones del tipo:
\begin{equation*}
    \frac{d^{2}f(x)}{dx^{2}} = a(x)f(x)
\end{equation*}

Sea $r$ un arreglo de números linealmente espaciados una distancia $h$.
\begin{itemize}
    \item $r_{n} = r_{n-1} + h$
\end{itemize}

Se introduce la notación:
\begin{gather*}
    u_{n}^{l} = u^{l}(r_{n}) = u^{l}(r_{n-1}+h) \\
    k_{n} = k(r_{n}) = k(r_{n-1}+h)
\end{gather*}

\end{frame}

\begin{frame}{}

De manera que:
\begin{equation*}
    u_{n+1}^{l} = u^{l}(r_{n}+h) \;\;\;\; u_{n-1}^{l} = u(r_{n}-h)
\end{equation*}

Por expansión en serie de Taylor alrededor de $r_{n}$:
\begin{gather}
    u_{n+1}^{l} = u(r_{n}) + hu'(r_{n}) + \frac{h^{2}}{2}u''(r_{n}) + \frac{h^{3}}{6}u'''(r_{n}) + \frac{h^{4}}{24}u^{(4)}(r_{n}) + O(h^{5}) \\
    u_{n-1}^{l} = u(r_{n}) - hu'(r_{n}) + \frac{h^{2}}{2}u''(r_{n}) - \frac{h^{3}}{6}u'''(r_{n}) + \frac{h^{4}}{24}u^{(4)}(r_{n}) + O(h^{5})
\end{gather}
    
\end{frame}

\begin{frame}{}

Aproximando $u''(r_{n})$ a la segunda derivada por diferencia de tres puntos:
\begin{equation}
    u''(r_{n}) = 
    \begin{pmatrix}
        \frac{2}{(r_{n-1}-r_{n-2})(r_{n}-r_{n-1})} \\[6pt] \frac{-2}{(r_{n}-r_{n-1})(r_{n-1}-r_{n-2})} \\[6pt] \frac{2}{(r_{n}-r_{n-1})(r_{n}-r_{n-2})}
    \end{pmatrix}^{T}
    \begin{pmatrix}
        u_{n-2} \\ u_{n-1} \\ u_{n}
    \end{pmatrix}
\end{equation}

Con (19)-(21):
\begin{equation}
    u_{n} = \frac{2\left(1-\frac{5h^{2}}{12}\right)K_{n-1}u_{n-1} - \left(1+\frac{h^{2}}{12}K_{n-2}\right)u_{n-2}}{\left(1+\frac{h^{2}}{12}K_{n}\right)}
\end{equation}
    
\end{frame}

\begin{frame}{}

\begin{equation}
    u_{n} = \frac{2\left(1-\frac{5h^{2}}{12}K_{n}\right)u_{n} - \left(1+\frac{h^{2}}{12}K_{n+1}\right)u_{n+1}}{\left(1+\frac{h^{2}}{12}K_{n-1}\right)}
\end{equation}

Las ecuaciones (22) y (23) son las relaciones recursivas hacia adelante y hacia atrás respectivamente.\\

Para calcular todos los $\{u_{i}\}=\{u_{0},u_{1},...,u_{n}\}$ necesitamos dos valores iniciales:
\begin{itemize}
    \item $u_{n-1}$ y $u_{n-2}$ si usamos la RRH Adelante
    \item $u_{n}$ y $u_{n+1}$ si usamos la RRH Atrás
\end{itemize}
    
\end{frame}

\begin{frame}{}

Y la derivada $u'_{n}$ queda dada por:
\begin{equation*}
    u'_{n} = \frac{1}{2h} \left[\left(1+\frac{h^{2}}{6}K_{n+1}\right)u_{n+1} - \left(1+\frac{h^{2}}{6}K_{n-1}\right)u_{n+1}\right] + O(h^{4})
\end{equation*}

Cómo determinar un eigenvalor de la energía?
\begin{itemize}
    \item Región $E>V(r) \rightarrow$ Clásica permitida
    \item Región $E<V(r) \rightarrow$ Clásica prohibida
\end{itemize}
\vspace{5mm}
Las regiones están separadas por un $r_{c}$ dado por:
\begin{equation*}
    E=V(r_{c})
\end{equation*}
    
\end{frame}

\begin{frame}

Sean:
\begin{equation*}
    u_{out} = \frac{2\left(1-\frac{5h^{2}}{12}K_{n-1}\right)u_{n-1} - \left(1+\frac{h^{2}}{12}K_{n-2}\right)u_{n-2}}{\left(1+\frac{h^{2}}{12}K_{n}\right)}
\end{equation*}
donde:
\begin{gather*}
    u_{n-1} = u(r_{c}) = u_{c} \\
    u_{n-2} = u(r_{c-1}) = u_{c-1}
\end{gather*}

\begin{equation*}
    u_{in} = \frac{2\left(1-\frac{5h^{2}}{12}K_{n}\right)u_{n} - \left(1+\frac{h^{2}}{12}K_{n-1}\right)u_{n+1}}{\left(1+\frac{h^{2}}{12}K_{n-1}\right)}
\end{equation*}
donde:
\begin{gather*}
    u_{n} = u(r_{c}) = u_{c} \\
    u_{n+1} = u(r_{c+1}) = u_{c+1}
\end{gather*}
    
\end{frame}

\begin{frame}{}

$E$ es un valor de la energía si:
\begin{equation*}
    G(E) = \left[\frac{u'_{out}}{u_{out}}\right]_{r=r_{c}} - \left[\frac{u'_{in}}{u_{in}}\right]_{r=r_{c}} = 0
\end{equation*}

Damos valores de $E$ tales que $E\in(E_{1},E_{2})$ y se calcula $G(E)$ hasta hallar un cambio de signo, supóngase entre $E_{j}$ y $E_{k}$.\\
\vspace{5mm}
$\Rightarrow$ Hay un eigenvalor de la energía entre $E_{j}$ y $E_{k}=E_{j+1}$
    
\end{frame}

\begin{frame}{Bottonium}

Para el bottonium debería encontrarse el primer eigenvalor de la energía aproximadamente en:
\begin{equation*}
    B^{15}_{est} = M^{15}_{exp}-2m_{b} = -0.8997 GeV
\end{equation*}
donde $M^{15}_{exp}=9.4603 GeV$ y $m_{b}=5.18 GeV$.\\
\vspace{5mm}
Para la búsqueda computacional el intervalo de energías será:
\begin{gather*}
    E_{if} = [-5,0] \\
    \Rightarrow B^{15}_{est} \in E_{if}
\end{gather*}
\vspace{5mm}
Se utiliza $\Delta E=0.5GeV$ y $len(E_{if})=11$.\\
Haciendo $B^{15}_{est}=V(r_{cut}) \Rightarrow r_{cut} \approx 0.1fm$.
    
\end{frame}

\begin{frame}{}

\begin{gather*}
    -\frac{4}{3}\frac{\alpha_{s}\hbar c}{r} +\sigma r = B^{15}_{est} \\
    0 = -\frac{4}{3}\frac{\alpha_{s}\hbar c}{r} - B^{15}_{est} + \sigma r \\
    r_{cut} \approx 0.10330565fm
\end{gather*}
\vspace{5mm}
Intervalo de exploración:
\begin{gather*}
    r_{max} = 1.0fm \\
    h = 0.001fm \\
    len(r) = 1000
\end{gather*}
    
\end{frame}

\begin{frame}{Condiciones de frontera}
\begin{gather*}
    u(r_{min}) = 0 \\
    u(r_{max}) = 0
\end{gather*}

Pero deben de ser dos desde $r\rightarrow r_{max}$.\\
Consideramos más general:
\begin{equation*}
    u(r\rightarrow 0) \rightarrow r^{l+1}
\end{equation*}
\begin{align*}
    u&(r\rightarrow \infty) \rightarrow exp\left[ -\frac{\sqrt{2\mu |E|}}{\hbar}r \right] \\
    &(r\rightarrow r_{max})
\end{align*}
    
\end{frame}