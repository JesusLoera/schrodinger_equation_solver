\section{La ecuación de Schrodinger unidimensional}


\begin{frame}{Ecuación de Schrodinger }

    La ecuación de Schrodinger  está dada por la siguiente ecuación.

    \begin{equation}
        i \hbar \frac{\partial \Psi }{\partial t} = \hat{H} \Psi
        \label{eqn:ecn_Schrodinger}
    \end{equation}

    Donde $\hat{H}$ es el operador hamiltoniano.

    \begin{equation}
        \hat{H} = -\frac{ \hbar^2 }{2m} \nabla^2 + V
        \label{eqn:op_hamiltoniano}
    \end{equation}

    Por lo tanto, al sustituir (\ref{eqn:op_hamiltoniano}) en (\ref{eqn:ecn_Schrodinger}).

    \begin{equation}
        i \hbar \frac{\partial \Psi }{\partial t}
        = 
        -\frac{ \hbar^2 }{2m} \nabla^2 \Psi + V \Psi
    \end{equation}

\end{frame}


\begin{frame}{Ecuación de Schrodinger unidimensional}

    Considerando un caso unidimensional, el laplaciano toma la siguiente forma.

    \begin{equation}
        \nabla^2 \Longrightarrow \frac{\partial^2 }{\partial x^2}
    \end{equation}

    
    Por lo tanto, la ecuación de Schrodinger unidimensional nos queda de la siguiente manera.

    \begin{equation}
        i \hbar \frac{\partial \Psi }{\partial t}
        = 
        -\frac{ \hbar^2 }{2m} \frac{\partial^2 \Psi}{\partial x^2} + V \Psi
        \label{eqn:ecn_Schrodinger_1d}
    \end{equation}

    Y la función de onda $\Psi$ debe satisfacer la condición de normalización.

    \begin{equation}
        \int \abs{ \Psi }^2 dV
        \label{eqn:cond_normalizacion_gral}
    \end{equation}

\end{frame}


\begin{frame}{Separación de variables}
    Observamos que la ecuación (\ref{eqn:ecn_Schrodinger_1d}) es una PDE, aplicaremos el método de separación de variables. 
    
    \vspace{0.2cm}
    
    Proponemos una solución de la forma:

    \begin{equation}
        \Psi(x, t) = \psi(x) T(t)
        \label{eqn:separacion}
    \end{equation}

    Al sustituir la ecuación (\ref{eqn:separacion}) en la ecuación (\ref{eqn:ecn_Schrodinger_1d}).

    \begin{equation}
        i \hbar \frac{1}{T(t)} \frac{dT(t)}{dt}
        =
        - \frac{\hbar^2}{2m} \frac{1}{\psi(x)} \frac{d^2\psi(x)}{dx^2} + V
        \label{eqn:ecn_separada}
    \end{equation}

    La parte izquierda de la ecuación es una función de t y la parte derecha es una función de x, solo pueden ser igualadas si son funciones constantes.

\end{frame}

\begin{frame}{Solución de la parte temporal}
    
    Sea E la constante de separación, entonces la igualamos con la parte izquierda de la
    ecuación (\ref{eqn:ecn_separada}).

    \begin{equation}
        i \hbar \frac{1}{T(t)} \frac{dT(t)}{dt}
        =
        E
    \end{equation}

    \begin{equation}
        i \hbar \frac{dT(t)}{dt}
        =
        E T(t)
    \end{equation}

    Cuya solución está dada por la siguiente expresión que es independiente del potencial $V$.

    \begin{equation}
        T(t) = e^{-iEt/\hbar}
    \end{equation}

\end{frame}


\begin{frame}{Ecuación de Schrodinger unidimensional independiente del tiempo}

    Y ahora igualamos la parte derecha de la ecuación (\ref{eqn:ecn_separada}) con E.

    \begin{equation}
        - \frac{\hbar^2}{2m} \frac{1}{\psi(x)} \frac{d^2\psi(x)}{dx^2} + V 
        =
        E
    \end{equation}

    De manera que nos queda la conocida \emph{ecuación de Schrodinger indenpendiente del tiempo}.

    \begin{equation}
        -\frac{\hbar^2}{2m} \frac{d^2}{dx^2} + V \psi = E \psi
        \label{eqn:ecn_Schrodinger_indp_tiempo}
    \end{equation}

    Esta ecuación tiene una solución distinta dependiendo del potencial
    de interacción $V$.

\end{frame}

\begin{frame}{Resumen}

    Los problemas unidimensionales con potenciales de la forma $V=V(x)$ pueden atacarse
    de la siguiente manera.

    \vspace{0.2cm}

    \begin{itemize}
        \item Determinar los eigenvalores de la energía E de la ecuación de 
        Schrodinger independiente del tiempo.
        \item Resolver para la función de onda independiente del tiempo $\psi$.
        \item La soluciones deben satisfacer la condición de normalizacion de la ecuación
        \ref{eqn:cond_normalizacion_gral}.
        \begin{equation*}
            \displaystyle\int_{-\infty}^{\infty} \abs{\psi}^2 dx
        \end{equation*}
    \end{itemize}

\end{frame}