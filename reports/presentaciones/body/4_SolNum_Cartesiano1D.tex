\begin{frame}{Soluciones numéricas}
    Para resolver la ecuación de Schrodinger de forma computacional haremos las siguientes consideraciones.
    \begin{enumerate}
        \item Como no podemos integrar la ecuación de Schrodinger de $-\infty$ a $\infty$, consideramos que nuestro sistema se encuentra confinado 
        \item Vamos a discretizar el espacio entre dos puntos, $x_0$ y $x_N$,en una serie de N+1 puntos equidistantes separados por una distancia $\Delta x$, es decir $ \left\{ x_0, x_1, x_2, ..., x_{N-1}, x_N\right\}$ tal que $x_i = x_o + (i-1)\Delta x$. 
        \item Introducimos la siguiente notación $ \psi_i \equiv  \psi(x_i)$ y $ V_i \equiv V(x_i)$.
    \end{enumerate}
\end{frame}


% TRIDIAGONAL MATRIX
\subsection{Método de matrices tridiagonales}


% ENERGY BISECTION
\subsection{Método de la bisección de energía}

\begin{frame}
    Primeramente veamos un método para resolver ecuaciones de la f
\end{frame}


% MATCHING METHOD

\subsection{Método de emparejamiento}

\begin{frame}{Método de emparejamiento}
    Vamos a resolver la ecuación de Schrodinger.
    \begin{equation}
        -\frac{\hbar^2}{2m} \psi^{''}(x) + V(x)\psi(x) = E\psi(x) 
    \end{equation}
    
    Sujeta a la siguientes condiciones de frontera.
    \begin{equation}
        \psi(x_0) = 0
        \quad \text{y} \quad
        \psi(x_N) = 0
    \end{equation}

    Usaremos el método del disparo para resolver la ecuación diferencial sujeta a condiciones en la frontera.
\end{frame}

\begin{frame}{Método de emparejamiento}
    El método del disparo para BVP  consiste en tratar el problema como un IVP y adivinar las condiciones iniciales que satisfagan las condiciones en la frontera. Resolveremos la ecn. de Schrodinger:

    \begin{equation}
        -\frac{\hbar^2}{2m} \psi^{''}(x) + V(x)\psi(x) = E\psi(x) 
        \label{eqn:schrodinger_1d_match}
    \end{equation}
    
    Sujeta a la siguientes condiciones iniciales.
    \begin{equation}
        \psi(x_0) = 0
        \quad \text{y} \quad
        \psi^{'}(x_o) = s
    \end{equation}

    Donde $s$ es un número real distinto de 0 y se varía hasta satisfacer las condiciones en la frontera, $\psi(x_0) = 0$ y $\psi(x_N) = 0$, se conoce como el \emph{parametro libre de disparo}.
\end{frame}

\begin{frame}{Método de emparejamiento}

    Dado un valor de la energía E, buscaremos una forma de integrar la ecuación de Schrodinger para determinar la función de onda $\psi$.
    
    \vspace{0.2cm}

    La segunda derivada de 3 puntos en un punto arbitrario $\psi_i$ está dada por la siguiente expresión.

    \begin{equation}
        \psi_{i}^{''} 
        =
        \frac{ \psi_{i+1} - 2\psi_{i} +  \psi_{i-1} }{ (\Delta x)^2 }
        ;\quad
        i= 1,2,...,N-1
        \label{eqn:2_derivada_3pts_match}
    \end{equation}

    Sustituyendo la ecuación (\ref{eqn:2_derivada_3pts_match}) en la ecuación (\ref{eqn:schrodinger_1d_match}).

    \begin{equation}
        -\frac{\hbar^2}{2m} \left[ 
            \frac{ \psi_{i+1} - 2\psi_{i} +  \psi_{i-1} }{ (\Delta x)^2 }
         \right] + V_i\psi_i 
         = 
         E\psi_i 
         ;\quad
        i= 1,...,N-1
         \label{eqn:discrete_schrodinger_matching}
    \end{equation}

\end{frame}

\begin{frame}{Método de emparejamiento}
    Despejando $\psi_{i+1}$ de la ecuación (\ref{eqn:discrete_schrodinger_matching}) tenemos la \emph{fórmula de integración hacia adelante}.

    \begin{equation}
        \psi_{i+1}
        =
        2\left[
            1+ m\frac{(\Delta)^2}{\hbar^2} (V_i - E) 
        \right] \psi_i - \psi_{i-1}
        ;\quad
        i= 1,...,N-1
        \label{eqn:integracion_adelante_matching}
    \end{equation}

    Despejando $\psi_{i-1}$ de la ecuación (\ref{eqn:discrete_schrodinger_matching}) tenemos la \emph{fórmula de integración hacia atrás}.

    \begin{equation}
        \psi_{i-1}
        =
        2\left[
            1+ m\frac{(\Delta)^2}{\hbar^2} (V_i - E) 
        \right] \psi_i - \psi_{i+1}
        ;\quad
        i= 1,...,N-1
        \label{eqn:integracion_atras_matching}
    \end{equation}

\end{frame}

\begin{frame}{Método de emparejamiento}
    Para integrar la ecuación de Schrodinger hacia delante necesitamos 2 valores iniciales, $\psi_0$ y $\psi_1$, pero por el momento desconocemos $\psi_1$.

    La primera derivada de dos puntos hacia adelante está dada por la siguiente ecuación.
    \begin{equation}
        \psi^{'}_{i} = \frac{\psi_{i+1} - \psi_{i} }{ \Delta x}
    \end{equation}

    Particularmente cuando $i=0$ tenemos que:
    \begin{equation}
        \psi^{'}_{0} = \frac{\psi_{1} - \psi_{} }{ \Delta x}
    \end{equation}

    Y recordando las condiciones iniciales $\psi^{'}_{0}=s$ y $\psi_{0}=0$ nos queda la siguiente ecuación.

    \begin{equation}
        \psi_{1} = s \Delta x
    \end{equation}


\end{frame}